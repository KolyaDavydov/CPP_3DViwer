\documentclass{article}

% Language setting
% Replace `english' with e.g. `spanish' to change the document language
\usepackage[russian]{babel}

% Set page size and margins
% Replace `letterpaper' with `a4paper' for UK/EU standard size
\usepackage[a4paper,top=2cm,bottom=2cm,left=3cm,right=3cm,marginparwidth=1.75cm]{geometry}

% Useful packages
\usepackage{amsmath}
\usepackage{graphicx}
\usepackage[colorlinks=true, allcolors=blue]{hyperref}

\title{3DViewer v1.0}
\author{marianel, ilanadah, nohoteth}

\begin{document}
\maketitle

\section{Введение}
3Д-обозреватель :
    \begin{itemize}
        \item Предназначен для просмотра 3D моделей в каркасном виде.
        \item Поддерживает вращение, масштабирование и перемещение моделей.
        \item Поддерживает следующие операции:
        \begin{enumerate}
            \item Загрузка каркасных моделей из файла формата obj (поддержка только списка вершин и поверхностей);
            \item Перемещение модели на заданное расстояние относительно осей X, Y, Z;
            \item Поворот модели на заданное расстояние относительно осей X, Y, Z;
            \item Масштабирование модели на заданное значение.
            \item Программа должна позволять настраивать тип проекции (параллельная и центральная);
            \item Программа позволяет настраивать тип (сплошная, пунктирная), цвет и толщину ребер, способ отображения (отсутствует, круг, квадрат), цвет и размер вершин;
            \item Программа позволяет выбирать цвет фона;
            \item Настройки должны сохраняться между перезапусками программы
        \end{enumerate}
        \item Программа позволяет сохранять полученные ("отрендеренные") изображения в файл в форматах bmp и jpeg
        \item Программа позволяет записывать небольшие "скринкасты" 
    \end{itemize}

\section{Установка}

\begin{enumerate}
    \item Скачайте репозиторий проекта;
    \item Перейдите в терминале в папку src проекта;
    \item Выполните \textbf{make install};
    \item Откройте папку приложения на рабочем столе и запустите приложение.
\end{enumerate}

\section{Удаление}

\begin{enumerate}
    \item Перейдите в терминале в папку src проекта;
    \item Выполните: \textbf{make uninstall};
    \item Удалите папку проекта.
\end{enumerate}

\end{document}
